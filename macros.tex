% Macros which I'd like to be shared/stable across projects.

\newcommand*{\lambdaCalculus}{$\lambda$-calculus\xspace}
\newcommand*{\lambdaCalculi}{$\lambda$-calculi\xspace}
\newcommand*{\piCalculus}{$\pi$-calculus\xspace}
\newcommand*{\piCalculi}{$\pi$-calculi\xspace}

% Misc.
\newcommand*{\etal}{et al\xspace}
\newcommand*{\ala}{\emph{\`a la}\xspace}

% Housekeeping.
\newcommand*{\todo}[1]{\textcolor{blue}{\textbf{#1}}}

% Formatting.
\newcommand*{\ttt}[1]{\texttt{#1}}
\newcommand*{\kw}[1]{{\text{\tt{#1}}}} % is this different from \ttt?

\newcommand*{\twoPrime}{{\prime\mkern-2.6mu\prime\mkern-2.2mu}}

% http://tex.stackexchange.com/questions/46690
\DeclareTextFontCommand{\textbfit}{%
  \fontseries\bfdefault % change series without selecting the font yet
  \itshape
}

% Misc math.
\newcommand*{\abs}[1]{|#1|}
\newcommand*{\after}{\circ}
\newcommand*{\disjoint}{\#}
\newcommand*{\disjunion}{\uplus}
\newcommand*{\Cdot}{\raisebox{-0.25ex}{\scalebox{1.4}{$\cdot$}}}
\newcommand*{\codom}[1]{\textsf{codom}{(#1)}}
\DeclareRobustCommand{\Concat}{\mathbin{\Cdot}} % http://tex.stackexchange.com/questions/272554
\newcommand*{\concat}{\cdot}
\newcommand*{\Dom}[1]{\textsf{dom}\;#1}
\newcommand*{\dom}[1]{\textsf{dom}(#1)}
\newcommand*{\domrestrict}[2]{#1{\mid_{#2}}}
\newcommand*{\EmptySet}{\varnothing}
\newcommand*{\eqdef}{\stackrel{\smash{\text{\tiny def}}}{=}}
\newcommand*{\family}[3]{\set{#1}_{\smash{#2\in#3}}}
\newcommand*{\iso}{\cong}
\newcommand*{\judge}{\Longrightarrow}
\newcommand*{\maps}[2]{#1 \mapsto #2}
\newcommand*{\maplet}[2]{\set{\maps{#1}{#2}}}
\newcommand*{\plusplus}{\ensuremath{\mathbin{+\mkern-10mu+}}}
\newcommand*{\ran}[1]{\textsf{ran}(#1)}
\newcommand*{\set}[1]{\{#1\}}
\newcommand*{\tensor}{\otimes}

% Projections.
\newcommand*{\projOne}[1]{\ensuremath{{\pi_1}\,#1}}
\newcommand*{\projTwo}[1]{\ensuremath{{\pi_2}\,#1}}

% Option/Maybe.
\newcommand*{\optSome}[1]{\textsf{some}\;{#1}}
\newcommand*{\optNone}[0]{\textsf{none}}

% Relational composition.
\DeclareSymbolFont{bbsymbol}{U}{bbold}{m}{n}
\DeclareMathSymbol{\bbsemi}{\mathbin}{bbsymbol}{"3B}

% Lattices.
\newcommand*{\join}{\sqcup}
\newcommand*{\meet}{\sqcap}
\newcommand*{\bigjoin}{\bigsqcup}
\newcommand*{\bigmeet}{\bigsqcap}

% Linear logic (\with doesn't work for me and \parr looks silly).
\newcommand*{\LLpar}{\rotatebox[origin=c]{180}{\LLwith}}
\newcommand*{\LLplus}{\oplus}
\newcommand*{\LLtimes}{\otimes}
\newcommand*{\LLwith}{\text{\&}}

% (Implicit?) indices
\newcommand*{\lowlight}[1]{\textcolor{darkgray}{#1}}
\newcommand*{\sub}[2]{#1_{\lowlight{#2}}}
\renewcommand*{\sup}[2]{#1^{\lowlight{#2}}}

\renewcommand*{\vdots}{\MnSymbolvdots}
\newcommand*{\param}{\cdot}

% Cross-referencing.
\newcommand*{\appref}[1]{Appendix~\ref{app:#1}}
\newcommand*{\conjref}[1]{Conjecture~\ref{conj:#1}}
\newcommand*{\corref}[1]{Corollary~\ref{cor:#1}}
\newcommand*{\defref}[1]{Definition~\ref{def:#1}}
\newcommand*{\Defref}[1]{Definition~\ref{def:#1} (\nameref{def:#1})}
\newcommand*{\DefrefShort}[1]{\ref{def:#1} (\nameref{def:#1})}
\newcommand*{\defrefTwo}[2]{Definitions~\ref{def:#1} and \ref{def:#2}}
\newcommand*{\eqnref}[1]{Equation~\ref{eqn:#1}}
\newcommand*{\eqnrefShort}[1]{Eqn.~\ref{eqn:#1}}
\newcommand*{\eqnrefTwo}[2]{Equations~\ref{eqn:#1} and \ref{eqn:#2}}
\newcommand*{\exref}[1]{Example~\ref{ex:#1}}
\newcommand*{\exrefTwo}[2]{Examples~\ref{ex:#1} and \ref{ex:#2}}
\newcommand*{\figref}[1]{Figure~\ref{fig:#1}}
\newcommand*{\figrefTwo}[2]{Figures \ref{fig:#1} and \ref{fig:#2}}
\newcommand*{\figrefThree}[3]{Figures \ref{fig:#1}, \ref{fig:#2} and \ref{fig:#3}}
\newcommand*{\lemref}[1]{Lemma~\ref{lem:#1}}
\newcommand*{\Lemref}[1]{Lemma~\ref{lem:#1} (\nameref{lem:#1})}
\newcommand*{\LemrefShort}[1]{\ref{lem:#1} (\nameref{lem:#1})}
\newcommand*{\lemrefTwo}[2]{Lemmas~\ref{lem:#1} and \ref{lem:#2}}
\newcommand*{\lemrefThree}[3]{Lemmas \ref{lem:#1}, \ref{lem:#2} and \ref{lem:#3}}
\newcommand*{\lemrefFour}[4]{Lemmas \ref{lem:#1}, \ref{lem:#2}, \ref{lem:#3} and \ref{lem:#4}}
\newcommand*{\lstref}[1]{Listing~\ref{lst:#1}}
\newcommand*{\secref}[1]{Section~\ref{sec:#1}}
\newcommand*{\secrefTwo}[2]{Sections~\ref{sec:#1} and \ref{sec:#2}}
\newcommand*{\Secref}[1]{\S\,\ref{sec:#1}}
\newcommand*{\SecrefTwo}[2]{\S\S\,\ref{sec:#1} and \ref{sec:#2}}
\newcommand*{\thmref}[1]{Theorem~\ref{thm:#1}}
\newcommand*{\Thmref}[1]{Theorem~\ref{thm:#1} (\nameref{thm:#1})}
\newcommand*{\thmrefTwo}[2]{Theorems~\ref{thm:#1} and \ref{thm:#2}}

% Environments.
\newenvironment{nop}{}{}
\newenvironment{sdisplaymath}
   {\begin{nop}\small\begin{displaymath}}
   {\end{displaymath}\end{nop}\ignorespacesafterend}
\newenvironment{smathpar}
   {\begin{nop}\small\begin{mathpar}}
   {\end{mathpar}\end{nop}\ignorespacesafterend}
\newenvironment{salign}
   {\par\nobreak\small\noindent\csname align*\endcsname}
   {\csname endalign*\endcsname}
\newenvironment{sqitemize}
    {\begin{itemize}[label=$\sqbullet$]}
    {\end{itemize}}

% Figures.
\newenvironment{mathfig}{\begin{sdisplaymath}}{\end{sdisplaymath}}
\newenvironment{syntaxfig}{\begin{mathfig}\begin{array}{@{}l@{\quad}r@{~~}c@{\quad}ll}}{\end{array}\end{mathfig}}

% Sub-floats.
\makeatletter
\newbox\sf@box
\newenvironment{SubFloat}[2][]%
  {\def\sf@one{#1}%
   \def\sf@two{#2}%
   \setbox\sf@box\hbox
      \bgroup}%
  { \egroup
   \ifx\@empty\sf@two\@empty\relax
     \def\sf@two{\@empty}
   \fi
   \ifx\@empty\sf@one\@empty\relax
      \subfloat[\sf@two]{\box\sf@box}%
   \else
      \subfloat[\sf@one][\sf@two]{\box\sf@box}%
   \fi}
\makeatother

\RequirePackage{xcolor}

% Shaded box.
\definecolor{highlightcolor}{rgb}{1.0,0.8,0.8}
\definecolor{shadecolor}{rgb}{0.9,0.9,0.9}
\definecolor{lightgray}{rgb}{0.8,0.8,0.8}
% Previous definition using \strut was somewhat unpredictable. See http://tex.stackexchange.com/questions/86569.
% \newcommand*{\shadebox}[1]{\fcolorbox{lightgray}{shadecolor}{\strut{#1}}}
\newcommand*{\shadebox}[1]{\fcolorbox{lightgray}{shadecolor}{\raisebox{0pt}[0.60\baselineskip][0.05\baselineskip]{#1}}}

% Inference rules.
\newenvironment{ruleblock}[1]{%
\begin{flushleft}\shadebox{$#1$}\end{flushleft}\begin{smathpar}}{\end{smathpar}%
}

\newcommand*{\ruleName}[1]{\textnormal{\textsf{#1}}}

% See http://tex.stackexchange.com/questions/171810/
\newcommand*\first[1]{\mathchoice{#1^\prime}{#1^\prime}{#1^\prime}%
   {#1^{\scalebox{.7}{$\scriptscriptstyle\prime$}}}
}

% https://tex.stackexchange.com/questions/21644.
\makeatletter
\newcommand{\superimpose}[2]{%
  {\ooalign{$#1\@firstoftwo#2$\cr\hfil$#1\@secondoftwo#2$\hfil\cr}}}
\makeatother

% This doesn't work properly. Use \NSCenter below instead.
% http://tex.stackexchange.com/questions/98839
\newenvironment{nscenter}
 {\parskip=0pt\par\nopagebreak\centering}
 {\par\noindent\ignorespacesafterend}

% https://tex.stackexchange.com/questions/24186
% Was previously called \Center but that clashes with ragged2e package.
\newcommand{\NSCenter}[1]{\makebox[\textwidth]{#1}}

\newcommand*{\xrightarrowdbl}[2][]{%
  \xrightarrow[#1]{#2}\mathrel{\mkern-15mu}\rightarrow
}

% http://tex.stackexchange.com/questions/295635
\newsavebox{\vardisplaymathbox}
\newenvironment{vardisplaymath}[1][\small]
  {\begin{displaymath}\begin{lrbox}{\vardisplaymathbox}
   #1$\displaystyle}
  {$\end{lrbox}\usebox{\vardisplaymathbox}\end{displaymath}%
    \ignorespacesafterend}

\newcommand{\crossrule}{\noindent\textcolor{lightgray}{\cleaders\hbox{.}\hfill}}
